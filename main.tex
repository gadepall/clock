\documentclass[11pt]{article}

% ---------------- Page & Font ----------------
\usepackage{fullpage}
\renewcommand{\familydefault}{\sfdefault}

% ---------------- Encoding & Language ----------------
\usepackage[utf8]{inputenc}
\usepackage[english]{babel}
\usepackage{CJKutf8}

% ---------------- Formatting ----------------
\usepackage{setspace}
\usepackage{parskip}
\usepackage{titlesec}
\usepackage[section]{placeins}
\usepackage{float}
\usepackage{lineno}
\usepackage{hyphenat}
\usepackage{xcolor}
\usepackage{breakcites}

% ---------------- Math & Symbols ----------------
\usepackage{amsmath}
\usepackage{amsfonts}
\usepackage{amssymb}

% ---------------- Figures & Tables ----------------
\usepackage{graphicx}
\usepackage[space]{grffile}
\usepackage{subcaption}
\usepackage{longtable}
\usepackage{tabulary}
\usepackage{booktabs,array,multirow}

% ---------------- Listings ----------------
\usepackage{listings}

% ---------------- Karnaugh Maps ----------------
\usepackage{karnaugh-map}

% ---------------- Hyperref ----------------
\PassOptionsToPackage{hyphens}{url}
\usepackage[colorlinks = true,
            linkcolor = blue,
            urlcolor  = blue,
            citecolor = blue,
            anchorcolor = blue]{hyperref}

% ---------------- Float Fix ----------------
\usepackage{etoolbox}
\makeatletter
%\patchcmd\@combinedblfloats{\box\@outputbox}{\unvbox\@outputbox}{}{%
%  \errmessage{\noexpand\@combinedblfloats could not be patched}%
%}
\makeatother

% ---------------- Citations ----------------
\usepackage[round]{natbib}
\let\cite\citep

% ---------------- Abstract Style ----------------
\renewenvironment{abstract}
  {{\bfseries\noindent Abstract\par}\footnotesize}
  {\bigskip}

% ---------------- Section Spacing ----------------
\titlespacing{\section}{0pt}{*3}{*1}
\titlespacing{\subsection}{0pt}{*2}{*0.5}
\titlespacing{\subsubsection}{0pt}{*1.5}{0pt}

% ---------------- Graphics Extensions ----------------
\AtBeginDocument{
\DeclareGraphicsExtensions{.pdf,.PDF,.eps,.EPS,.png,.PNG,.jpg,.JPG,.jpeg,.JPEG}
}

% ---------------- Document ----------------
\begin{document}
\begin{CJK}{UTF8}{gbsn}

% ---------------- Title ----------------
\title{\textbf{Digital Clock using the Arduino Framework}}

\author{
Dhawal Saini \\[0.3em]
\textit{Department of Electrical Engineering} \\
\textit{Indian Institute of Technology Hyderabad}
\and
G. V. V. Sharma \\[0.3em]
\textit{Department of Electrical Engineering} \\
\textit{Indian Institute of Technology Hyderabad} \\
\texttt{gadepall@ee.iith.ac.in}
}

\date{\today}
%\date{January 08, 2026}

\begingroup
\let\center\flushleft
\let\endcenter\endflushleft
\maketitle
\endgroup

% ---------------- Abstract ----------------
\begin{abstract}
In this paper the design and implementation of a feature-rich digital clock is demonstrated. The system uses multiplexing to drive six seven-segment displays efficiently, minimizing I/O utilization. Key functionalities include timekeeping, digit-by-digit editing, and pause/play control. Boolean-based increment and decrement logic ensures more accurate cascading of seconds, minutes, and hours within standard constraints. The hardware setup, complemented by software debouncing and display refreshing, demonstrates a reliable, compact, and user-interactive digital clock suitable for both educational and practical applications.
\end{abstract}

% ---------------- Figure ----------------
\begin{figure}[H]
\centering
\includegraphics[width=0.9\columnwidth]{figs/Clock_Tinkercad.png}
\caption{Tinkercad Simulation of the Digital Clock}
\label{fig:tinker}
\end{figure}

% ---------------- Sections ----------------
\section{Introduction}
Digital timekeeping has long been a critical component of electronic system design, with classical digital design principles thoroughly discussed in foundational works such as \cite{mano2013digital, malvino2017digital, patterson2014computer}. The advent of microcontroller platforms, particularly Arduino, has enabled the development of compact, programmable clocks with enhanced user interactivity \cite{arduino_reference}. Techniques such as BCD-to-seven-segment interfacing and display multiplexing allow efficient utilization of limited I/O resources while maintaining accurate visual representation \cite{ti7447datasheet}. Inspired by these principles, a simple state machine representing a decade counter is implemented in \cite{ddta}. Based on this, we design an Arduino-based digital clock featuring six-digit multiplexed displays, pause/play functionality, and digit-by-digit editing with Boolean logic-driven increment and decrement operations. Most digital clock designs rely on standard counters as the building blocks.  However, our design is simpler since it relies on the state machine in \cite{ddta}.

\section{Clock Functionality and Hardware Setup}
Fig.~\ref{fig:tinker} shows the various hardware connections and the corresponding components are listed in Table~\ref{tab:component}. The clock follows the 23:59:59 format using six displays. The first two represent hours, the next two denote minutes, and the last two count the seconds. The six displays are connected to the Arduino through the pin connections listed in Table~\ref{tab:2.3}. The number count for each display is also listed therein.

\begin{table}[H]
\centering
\begin{tabular}{|c|c|c|}
\hline
Display & Arduino Pin & Description\\
\hline
1  & D4 & Hours Tens Digit\\
\hline
2  & D5 & Hours Units Digit\\
\hline
3  & D6 & Minutes Tens Digit\\
\hline
4  & D7 & Minutes Units Digit\\
\hline
5 & D8 & Seconds Tens Digit\\
\hline
6  & D9 & Seconds Units Digit\\
\hline
\end{tabular}

\caption{Display to Arduino Connections}
\label{tab:2.3}
\end{table}

Pin connections from the Arduino to the four push buttons are available in Table~\ref{tab:2.1}. Table~\ref{tab:2.2} gives the connections between the Arduino and the IC 7447 display decoder. Further, the pin connections for the IC 7447 to all the displays are available in Table~\ref{tab:3}.

\begin{table}[H]
\centering
\centering
\begin{tabular}{|l|c|c|}
\hline
Component & Value & Quantity\\
\hline
Arduino Uno & & 1\\
\hline
USB Cable & Type B & 1\\
\hline
Seven Segment Display & Common Cathode & 6\\
\hline
Push Buttons & & 4\\
\hline
IC 7447  &  & 1\\
\hline
Jumper Wires & M-M & 16\\
\hline
Breadboard & & 1\\
\hline
Resistors & 220$\Omega$ & 7\\
\hline
Resistors & 10$k\Omega$ (pull-down) & 4\\
\hline
\end{tabular}\\
\centerline{Table 1.0: Components List}

\caption{Components List}
\label{tab:component}
\end{table}

\begin{table}[H]
\centering
\centering
\begin{tabular}{|c|c|c|}
\hline
Button & Arduino Pin & Description\\
\hline
1 & D10 & Edit Mode Toggle\\
\hline
2 & D11 & Next Digit Selection\\
\hline
3 & D12  & Increment Digit\\
\hline
4 & D13  & Decrement Digit\\
\hline

\end{tabular}

\caption{Button to Arduino Connections}
\label{tab:2.1}
\end{table}

\begin{table}[H]
\centering
\centering
\begin{tabular}{|c|c|c|}
\hline
IC 7447 PIN & Arduino Pin & Description\\
\hline

7 & D0 & BCD Bit 0 (A)\\
\hline
1 & D1 & BCD Bit 1 (B)\\
\hline
2 & D2 & BCD Bit 2 (C)\\
\hline
6 & D3 & BCD Bit 3 (D)\\
\hline
\end{tabular}\\
\centerline{Table 2.2: IC 7447 to Arduino Connections}

\caption{IC 7447 to Arduino Connections}
\label{tab:2.2}
\end{table}

\begin{table}[H]
\centering
\centering
\begin{tabular}{|c|c|c|}
\hline
IC 7447 & Seven Segment (All) & Name\\
\hline
Pin 13 & a & Controls segment a\\
\hline
Pin 12 & b & Controls segment b\\
\hline
Pin 11 & c & Controls segment c\\
\hline
Pin 10 & d & Controls segment d\\
\hline
Pin 9 & e & Controls segment e\\
\hline
Pin 15 & f & Controls segment f\\
\hline
Pin 14 & g & Controls segment g\\
\hline
Pin 8 & Ground & Ground Supply\\
\hline
Pin 16 & 5V & Power Supply\\
\hline
\end{tabular}\\
\centerline{Table 3.0: BCD to 7-Segment Connections}

\caption{BCD to 7-Segment Connections}
\label{tab:3}
\end{table}


\section{Clock Logic}
Here, we discuss how to design counters for the various displays in the clock.
%\begin{figure}[H]
%\centering
%\includegraphics[width=0.5\textwidth]{figs/clock.jpg}
%\caption{Final Arduino-based Clock Implementation}
%\end{figure}

%\section{Increment Logic and Truth Tables}

%\subsection{Seconds Ones (0-9)}

\subsection{Displays 2 (Display1 $\ne 2$), 4 and 6}
These displays count from 0-9.  Display 2 will do this only when Display 1 is less than 2.
The truth table 
%is available in 
%Table 
%\ref{tab:4}
and the corresponding K-maps using dont cares for the output variables are given in 
Fig. \ref{fig:0-9}
along with the corresponding logic equations.
%Figs. \ref{fig:kmapA-4-6}-\ref{fig:kmapD-4-6} 
\subsection{Displays 3 and 5}
These displays count from 0-5.  
The truth table 
%is available in 
%Table 
%\ref{tab:5}
and the corresponding K-maps using dont cares for the output variables are given in 
Fig. \ref{fig:0-5},
%Figs. \ref{fig:kmapA-3-5}-\ref{fig:kmapC-3-5} 
along with the corresponding logic equations.
    $D = 0$ in this case.
\begin{figure*}[p]
    \centering

    % ------------------ Truth Table ------------------
    \begin{center}
\begin{tabular}{|c|c|c|c||c|c|c|c|}
\hline
Z & Y & X & W & D & C & B & A\\
\hline
0 & 0 & 0 & 0 & 0 & 0 & 0 & 1\\
0 & 0 & 0 & 1 & 0 & 0 & 1 & 0\\
0 & 0 & 1 & 0 & 0 & 0 & 1 & 1\\
0 & 0 & 1 & 1 & 0 & 1 & 0 & 0\\
0 & 1 & 0 & 0 & 0 & 1 & 0 & 1\\
0 & 1 & 0 & 1 & 0 & 0 & 0 & 0\\
\hline
\end{tabular}
\end{center}


    % ------------------ K-Maps (1x3 layout) ------------------
    \vspace{0.5cm}

    \begin{subfigure}[b]{0.3\textwidth}
        \centering
        \begin{karnaugh-map}[4][4][1][][]
            \maxterms{1,3,5}
            \minterms{0,2,4}
            \indeterminants{6,7,8,9,10,11,12,13,14,15}

            \implicantedge{0}{8}{2}{10}
            \draw[color=black, ultra thin] (0, 4) --
                node [pos=0.7, above right, anchor=south west] {$XW$}
                node [pos=0.7, below left, anchor=north east] {$ZY$} 
                ++(135:1);
        \end{karnaugh-map}
        \subcaption*{(A)}
        \label{fig:kmapA-3-5}
    \end{subfigure}
    \hfill
    \begin{subfigure}[b]{0.3\textwidth}
        \centering
        \begin{karnaugh-map}[4][4][1][][]
            \maxterms{0,4,3,5}
            \minterms{1,2}
            \indeterminants{6,7,8,9,10,11,12,13,14,15}

            \implicant{2}{10}
            \implicantedge{1}{1}{9}{9}

            \draw[color=black, ultra thin] (0, 4) --
                node [pos=0.7, above right, anchor=south west] {$XW$}
                node [pos=0.7, below left, anchor=north east] {$ZY$} 
                ++(135:1);
        \end{karnaugh-map}
        \subcaption*{(B)}
        \label{fig:kmapB-3-5}
    \end{subfigure}
    \hfill
    \begin{subfigure}[b]{0.3\textwidth}
        \centering
        \begin{karnaugh-map}[4][4][1][][]
            \maxterms{0,1,2,5}
            \minterms{3,4}
            \indeterminants{6,7,8,9,10,11,12,13,14,15}

            \implicant{4}{8}
            \implicant{3}{11}

            \draw[color=black, ultra thin] (0, 4) --
                node [pos=0.7, above right, anchor=south west] {$XW$}
                node [pos=0.7, below left, anchor=north east] {$ZY$} 
                ++(135:1);
        \end{karnaugh-map}
        \subcaption*{(C)}
        \label{fig:kmapC-3-5}
    \end{subfigure}

    \caption{Counting 0-5}
    \label{fig:0-5}
\end{figure*}


\subsection{Display 2 (Display1 = 2)}
This display counts from 0-3, when the first display shows 2.  
The truth table 
%is available in 
%Table 
%\ref{tab:6}
and the corresponding K-maps using dont cares for the output variables are given in 
Fig. \ref{fig:0-3}.
along with the corresponding logic equations.
    $C=D = 0$ in this case.
%Figs. \ref{fig:kmapA-2}-\ref{fig:kmapB-2} 
%
\begin{figure*}[p]
	\centering
	%\begin{table}[!h]
%\centering
\input{tables/logic/table6.tex}
%\caption{Counting 0-3}
%\captionof{table}{Counting 0-3}
%\label{tab:6}
%\end{table}
%	\begin{subfigure}[b]{0.45\textwidth}
%	\centering
%\begin{karnaugh-map}[2][2][1][][]
%    \minterms{0}
%    \maxterms{2,1}
%    \indeterminants{3}
%    \implicant{0}{0}
%    \draw[color=black, ultra thin] (0, 2) --
%        node [pos=0.7, above right, anchor=south west] {$W$}
%        node [pos=0.7, below left, anchor=north east] {$X$} 
%        ++(135:1);
%\end{karnaugh-map}
%\caption{A}
%\label{fig:kmapA-2}
%\end{subfigure}
%	\begin{subfigure}[b]{0.45\textwidth}
%	\centering
%\begin{karnaugh-map}[2][2][1][][]
%    \minterms{0}
%    \maxterms{2,1}
%    \indeterminants{3}
%    \implicant{0}{0}
%    \draw[color=black, ultra thin] (0, 2) --
%        node [pos=0.7, above right, anchor=south west] {$W$}
%        node [pos=0.7, below left, anchor=north east] {$X$} 
%        ++(135:1);
%\end{karnaugh-map}
%\caption{A}
%\label{fig:kmapA-2}
%\end{subfigure}
 \begin{subfigure}[b]{0.45\textwidth}
	\centering
\begin{karnaugh-map}[2][2][1][][]
    \minterms{0,2}
    \maxterms{3,1}

    \implicant{0}{2}
    \draw[color=black, ultra thin] (0, 2) --
        node [pos=0.7, above right, anchor=south west] {$W$}
        node [pos=0.7, below left, anchor=north east] {$X$} 
        ++(135:1);
\end{karnaugh-map}
\subcaption{$A = W_5^{\prime}$}
\label{fig:kmapA-2}
\end{subfigure}
 \begin{subfigure}[b]{0.45\textwidth}
	\centering
\begin{karnaugh-map}[2][2][1][][]
    \minterms{1,2}
    \maxterms{3,0}
    \implicant{1}{1}
    \implicant{2}{2}
    \draw[color=black, ultra thin] (0, 2) --
        node [pos=0.7, above right, anchor=south west] {$W$}
        node [pos=0.7, below left, anchor=north east] {$X$} 
        ++(135:1);
\end{karnaugh-map}
\subcaption{$B = W_5 X_5^{\prime} + W_5^{\prime} X_5$}
\label{fig:kmapB-2}
\end{subfigure}
\caption{Counting 0-3}
\label{fig:0-3}
\end{figure*}

\subsection{Display 1}
This display counts from 0-2, representing the first digit of the hour. 
The truth table 
%is available in 
%Table 
%\ref{tab:7}
and the corresponding K-maps using dont cares for the output variables are given in 
%Figs. \ref{fig:kmapA-2le2}-\ref{fig:kmapB-2le2}
Fig. \ref{fig:0-2},
along with the corresponding logic equations.
    $C=D = 0$ in this case.
%
\begin{figure*}[t]
	\centering
%\begin{table}[!h]
%\centering
\input{tables/logic/table7.tex}
%\caption{}
%\label{tab:7}
%\end{table}
	\begin{subfigure}[b]{0.45\textwidth}
	\centering
\begin{karnaugh-map}[2][2][1][][]
    \minterms{0}
    \maxterms{2,1}
    \indeterminants{3}
    \implicant{0}{0}
    \draw[color=black, ultra thin] (0, 2) --
        node [pos=0.7, above right, anchor=south west] {$W$}
        node [pos=0.7, below left, anchor=north east] {$X$} 
        ++(135:1);
\end{karnaugh-map}
\caption{A}
\label{fig:kmapA-2le2}
\end{subfigure}
%%
%\begin{figure}
%	\centering
\hfill
	\begin{subfigure}[b]{0.45\textwidth}
	\centering
\begin{karnaugh-map}[2][2][1][][]
    \minterms{1}
    \maxterms{2,0}
    \indeterminants{3}
    \implicant{1}{1}
    \draw[color=black, ultra thin] (0, 2) --
        node [pos=0.7, above right, anchor=south west] {$W$}
        node [pos=0.7, below left, anchor=north east] {$X$} 
        ++(135:1);
\end{karnaugh-map}
\caption{B}
\label{fig:kmapB-2le2}
\end{subfigure}
\caption{Counting 0-2}
\label{fig:0-2}
\end{figure*}


\section{Control Logic and Implementation}
\section{Decrement Logic}

\subsection{Seconds Ones (0-9)}
\input{decrement-logic/tables/table8.tex}

\begin{karnaugh-map}[4][4][1][][]
    \minterms{0,2,4,6,8}
    \maxterms{1,3,5,7,9}
    \indeterminants{10,11,12,13,14,15}

    \implicantedge{0}{8}{2}{10}
    \draw[color=black, ultra thin] (0, 4) --
        node [pos=0.7, above right, anchor=south west] {$XW$}
        node [pos=0.7, below left, anchor=north east] {$ZY$} 
        ++(135:1);
\end{karnaugh-map}
\begin{align}
    A &= W_1' \notag
\end{align}

\begin{karnaugh-map}[4][4][1][][]
    \minterms{3,4,7,8}
    \maxterms{0,1,2,5,6,9}
    \indeterminants{10,11,12,13,14,15}

    \implicant{3}{7}
    \implicant{4}{4}
    \implicant{8}{8}
    
    \draw[color=black, ultra thin] (0, 4) --
        node [pos=0.7, above right, anchor=south west] {$XW$}
        node [pos=0.7, below left, anchor=north east] {$ZY$} 
        ++(135:1);
\end{karnaugh-map}
\begin{align}
    B &= (X_1' W_1' ((Z_1' Y_1) + (Z_1 Y_1'))) + (Z_1' W_1 X_1) \notag
\end{align}

\begin{karnaugh-map}[4][4][1][][]
    \minterms{5,6,7,8}
    \maxterms{0,1,2,3,4,9}
    \indeterminants{10,11,12,13,14,15}

    \implicant{5}{7}
    \implicant{7}{6}
    \implicant{8}{8}
    \draw[color=black, ultra thin] (0, 4) --
        node [pos=0.7, above right, anchor=south west] {$XW$}
        node [pos=0.7, below left, anchor=north east] {$ZY$} 
        ++(135:1);
\end{karnaugh-map}
\begin{align}
    C &= (Z_1' Y_1 (X_1 + W_1)) + (Z_1 X_1' W_1' Y_1') \notag
\end{align}

\begin{karnaugh-map}[4][4][1][][]
    \minterms{0,9}
    \maxterms{1,2,3,4,5,6,7,8}
    \indeterminants{10,11,12,13,14,15}

    \implicant{0}{0}
    \implicant{9}{9}
    \draw[color=black, ultra thin] (0, 4) --
        node [pos=0.7, above right, anchor=south west] {$XW$}
        node [pos=0.7, below left, anchor=north east] {$ZY$} 
        ++(135:1);
\end{karnaugh-map}
\begin{align}
    D &= X_1' Y_1' ((Z_1 W_1) + (Z_1' W_1')) \notag
\end{align}

\subsection{Seconds Tens (0-5)}
\input{decrement-logic/tables/table9.tex}

\begin{karnaugh-map}[4][4][1][][]
    \minterms{0,2,4}
    \maxterms{1,3,5}
    \indeterminants{6,7,8,9,10,11,12,13,14,15}

    \implicantedge{0}{8}{2}{10}
    \draw[color=black, ultra thin] (0, 4) --
        node [pos=0.7, above right, anchor=south west] {$XW$}
        node [pos=0.7, below left, anchor=north east] {$ZY$} 
        ++(135:1);
\end{karnaugh-map}
\begin{align}
    A &= W_2' \notag
\end{align}

\begin{karnaugh-map}[4][4][1][][]
    \minterms{3,4}
    \maxterms{0,1,2,5}
    \indeterminants{6,7,8,9,10,11,12,13,14,15}

    \implicantedge{3}{3}{11}{11}
    \implicant{4}{12}

    \draw[color=black, ultra thin] (0, 4) --
        node [pos=0.7, above right, anchor=south west] {$XW$}
        node [pos=0.7, below left, anchor=north east] {$ZY$} 
        ++(135:1);
\end{karnaugh-map}
\begin{align}
    B &= (Y_2 X_2' W_2') + (Y_2' X_2 W_2) \notag
\end{align}

\begin{karnaugh-map}[4][4][1][][]
    \minterms{0,5}
    \maxterms{4,1,2,3}
    \indeterminants{6,7,8,9,10,11,12,13,14,15}

    \implicantedge{0}{0}{8}{8}
    \implicant{5}{13}

    \draw[color=black, ultra thin] (0, 4) --
        node [pos=0.7, above right, anchor=south west] {$XW$}
        node [pos=0.7, below left, anchor=north east] {$ZY$} 
        ++(135:1);
\end{karnaugh-map}
\begin{align}
    C &= X_2' ((Y_2 W_2) + (Y_2' W_2')) \notag
\end{align}

\begin{align}
    D &= 0 \notag
\end{align}

\subsection{Minutes Ones (0-9)}
\textit{Same as Seconds Ones with W3/X3/Y3/Z3.}

\subsection{Minutes Tens (0-5)}
\textit{Same as Seconds Tens with W4/X4/Y4/Z4.}

\subsection{Hours Ones }
\textbf{I. Tens = 0/1 → 0-9 }\\
\textit{Same as Seconds Ones with W5/X5/Y5/Z5.}

\textbf{II. Tens = 2 → 0-3 }
\input{decrement-logic/tables/table10.tex}

\begin{karnaugh-map}[2][2][1][][]
    \minterms{0,2}
    \maxterms{1,3}

    \implicant{0}{2}
    \draw[color=black, ultra thin] (0, 2) --
        node [pos=0.7, above right, anchor=south west] {$W$}
        node [pos=0.7, below left, anchor=north east] {$X$} 
        ++(135:1);
\end{karnaugh-map}
\begin{align}
    A &= W_5' \notag
\end{align}

\begin{karnaugh-map}[2][2][1][][]
    \minterms{0,3}
    \maxterms{1,2}

    \implicant{0}{0}
    \implicant{3}{3}

    \draw[color=black, ultra thin] (0, 2) --
        node [pos=0.7, above right, anchor=south west] {$W$}
        node [pos=0.7, below left, anchor=north east] {$X$} 
        ++(135:1);
\end{karnaugh-map}
\begin{align}
    B &= (X_5 W_5) + (X_5' W_5') \notag
\end{align}

\begin{align}
    C &= 0 \notag\\
    D &= 0 \notag
\end{align}

\subsection{Hours Tens (0-2)}
\input{decrement-logic/tables/table11.tex}

\begin{karnaugh-map}[2][2][1][][]
    \minterms{2}
    \maxterms{0,1}
    \indeterminants{3}

    \implicant{2}{2}
    \draw[color=black, ultra thin] (0, 2) --
        node [pos=0.7, above right, anchor=south west] {$W$}
        node [pos=0.7, below left, anchor=north east] {$X$} 
        ++(135:1);
\end{karnaugh-map}
\begin{align}
    A &= X_6 W_6' \notag
\end{align}

\begin{karnaugh-map}[2][2][1][][]
    \minterms{0}
    \maxterms{2,1}
    \indeterminants{3}

    \implicant{0}{0}
    \draw[color=black, ultra thin] (0, 2) --
        node [pos=0.7, above right, anchor=south west] {$W$}
        node [pos=0.7, below left, anchor=north east] {$X$} 
        ++(135:1);
\end{karnaugh-map}
\begin{align}
    B &= X_6' W_6' \notag
\end{align}

\begin{align}
    C &= 0 \notag\\
    D &= 0 \notag
\end{align}







\section{Multiplexing Technique}
All BCD inputs (A-D) are shared among six seven-segment displays. Displays are enabled one at a time using EN[0..5] = D4-D9. Each digit is displayed for 1ms, creating a fast alternating effect that appears continuous. This saves I/O pins and allows full six-digit display.


\section{Digit Editing Logic}
The clock allows pausing and digit-by-digit editing:

\begin{enumerate}
    \item Press PAUSE (D10) to toggle run/edit mode. In edit mode, the clock stops.
    \item Press NEXT (D11) to select the digit to edit (cycles 0-5: sec1, sec10, min1, min10, hr1, hr10).
    \item Press INC (D12) to increment the selected digit with rollovers.
    \item Press DEC (D13) to decrement the selected digit with rollunders.
    \item Selected digit blinks every 500ms to indicate focus.
\end{enumerate}


\subsection{Implementation}
    Pressing Button 1 toggles between run mode and edit mode. In edit mode, the clock pauses.  Following functions are available in edit mode.
    The selected digit blinks at 5Hz (200ms on, 200ms off) for visual feedback.
\begin{enumerate}
    \item Pressing Button 2 selects the next digit for editing (cycles through all six digits).
    \item Pressing Button 3 increments the currently selected digit using the increment logic tables.
    \item Pressing Button 4 decrements the currently selected digit using the decrement logic tables.
\end{enumerate}


\section{Software}
\section{Software Implementation}
The Arduino code implements:
\begin{itemize}
    \item Timer interrupt for clock ticking (10Hz interrupt rate)
    \item Button debouncing with software delays
    \item Multiplexed display refresh
    \item Editing mode with digit selection and value modification using the Boolean logic from the tables
    \item Proper constraints on time values (hours 0-23, minutes 0-59, seconds 0-59)
\end{itemize}


\section{Future Work}
The clock prototype on the Arduino is the first step towards building a chip.  The next step is to adapt the current design for implementation on a Vaman FPGA with a QuickLogic SoC and use additional EDA tools for physical chip design.  This will be done in future work.   
\iffalse
\begin{itemize}
\item Integration with wireless modules (Bluetooth/Wi-Fi) for remote time setting and synchronization.
\item Addition of alarms, timers, and countdown features with user-defined events.
\item Implementation of a real-time clock (RTC) module for improved accuracy and power efficiency.
\item Expansion to a multi-language or multi-format (12/24-hour) display interface.
\item Incorporation of IoT functionality for smart home or wearable applications.
\end{itemize}
\fi


% ---------------- Bibliography ----------------
\FloatBarrier
\bibliographystyle{plainnat}
\bibliography{references/references}

\end{CJK}
\end{document}

