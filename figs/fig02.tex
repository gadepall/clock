\begin{figure*}[t]
    \centering
    % --- Table first ---
    \input{tables/logic/table7.tex}
    \vspace{0.5cm} % spacing between table and Kmaps

    % --- K-map subfigures side by side ---
    \begin{subfigure}[b]{0.45\textwidth}
        \centering
        \begin{karnaugh-map}[2][2][1][][]
            \minterms{0}
            \maxterms{2,1}
            \indeterminants{3}
            \implicant{0}{0}
            \draw[color=black, ultra thin] (0, 2) --
                node [pos=0.7, above right, anchor=south west] {$W$}
                node [pos=0.7, below left, anchor=north east] {$X$} 
                ++(135:1);
        \end{karnaugh-map}
        \caption{A}
        \label{fig:kmapA-2le2}
    \end{subfigure}
    \hfill
    \begin{subfigure}[b]{0.45\textwidth}
        \centering
        \begin{karnaugh-map}[2][2][1][][]
            \minterms{1}
            \maxterms{2,0}
            \indeterminants{3}
            \implicant{1}{1}
            \draw[color=black, ultra thin] (0, 2) --
                node [pos=0.7, above right, anchor=south west] {$W$}
                node [pos=0.7, below left, anchor=north east] {$X$} 
                ++(135:1);
        \end{karnaugh-map}
        \caption{B}
        \label{fig:kmapB-2le2}
    \end{subfigure}

    % --- Main caption for entire figure ---
    \caption{Counting 0–2}
    \label{fig:0-2}
\end{figure*}

