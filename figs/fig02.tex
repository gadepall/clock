\begin{figure*}[t]
    \centering
    % --- Table first ---
    \begin{center}
\begin{tabular}{|c|c||c|c|c|c|}
\hline
X & W & D & C & B & A\\
\hline
0 & 0 & 0 & 0 & 0 & 1\\
0 & 1 & 0 & 0 & 1 & 0\\
1 & 0 & 0 & 0 & 0 & 0\\
\hline
\end{tabular}
\end{center}

    \vspace{0.5cm} % spacing between table and Kmaps

    % --- K-map subfigures side by side ---
    \begin{subfigure}[b]{0.45\textwidth}
        \centering
        \begin{karnaugh-map}[2][2][1][][]
            \minterms{0}
            \maxterms{2,1}
            \indeterminants{3}
            \implicant{0}{0}
            \draw[color=black, ultra thin] (0, 2) --
                node [pos=0.7, above right, anchor=south west] {$W$}
                node [pos=0.7, below left, anchor=north east] {$X$} 
                ++(135:1);
        \end{karnaugh-map}
        \subcaption{$A = W_6^{\prime} X_6^{\prime}$}
        \label{fig:kmapA-2le2}
    \end{subfigure}
    \hfill
    \begin{subfigure}[b]{0.45\textwidth}
        \centering
        \begin{karnaugh-map}[2][2][1][][]
            \minterms{1}
            \maxterms{2,0}
            \indeterminants{3}
            \implicant{1}{1}
            \draw[color=black, ultra thin] (0, 2) --
                node [pos=0.7, above right, anchor=south west] {$W$}
                node [pos=0.7, below left, anchor=north east] {$X$} 
                ++(135:1);
        \end{karnaugh-map}
        \subcaption{$B = W_6 X_6^{\prime}$}
        \label{fig:kmapB-2le2}
    \end{subfigure}

    % --- Main caption for entire figure ---
    \caption{Counting 0–2}
    \label{fig:0-2}
\end{figure*}

