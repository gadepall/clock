%\section{Increment Logic and Truth Tables}

%\subsection{Seconds Ones (0-9)}

\subsection{Displays 2 (Display1 $\ne 2$), 4 and 6}
These displays count from 0-9.  Display 2 will do this only when Display 1 is less than 2.
The truth table 
%is available in 
%Table 
%\ref{tab:4}
and the corresponding K-maps using dont cares for the output variables are given in 
Fig. \ref{fig:0-9}
along with the corresponding logic equations.
\begin{figure*}[t]
    \centering
%\begin{table}[!h]
%\centering
\begin{tabular}{|c|c|c|c||c|c|c|c|}
\hline
Z & Y & X & W & D & C & B & A\\
\hline
0 & 0 & 0 & 0 & 0 & 0 & 0 & 1\\
0 & 0 & 0 & 1 & 0 & 0 & 1 & 0\\
0 & 0 & 1 & 0 & 0 & 0 & 1 & 1\\
0 & 0 & 1 & 1 & 0 & 1 & 0 & 0\\
0 & 1 & 0 & 0 & 0 & 1 & 0 & 1\\
0 & 1 & 0 & 1 & 0 & 1 & 1 & 0\\
0 & 1 & 1 & 0 & 0 & 1 & 1 & 1\\
0 & 1 & 1 & 1 & 1 & 0 & 0 & 0\\
1 & 0 & 0 & 0 & 1 & 0 & 0 & 1\\
1 & 0 & 0 & 1 & 0 & 0 & 0 & 0\\
\hline
\end{tabular}

%\caption{Counting 0-9}
%\label{tab:4}
%\end{table}
\begin{subfigure}[b]{0.3\textwidth}
	\centering
\begin{karnaugh-map}[4][4][1][][]
    \maxterms{1,3,5,7,9}
    \minterms{0,2,4,6,8}
    \indeterminants{10,11,12,13,14,15}

    \implicantedge{0}{8}{2}{10}
    \draw[color=black, ultra thin] (0, 4) --
        node [pos=0.7, above right, anchor=south west] {$XW$}
        node [pos=0.7, below left, anchor=north east] {$ZY$} 
        ++(135:1);
\end{karnaugh-map}
\caption{A}
\label{fig:kmapA-4-6}
\end{subfigure}
\hfill
\begin{subfigure}[b]{0.3\textwidth}
	\centering
\begin{karnaugh-map}[4][4][1][][]
    \minterms{1,2,5,6}
    \maxterms{0,3,4,7,8,9}
    \indeterminants{10,11,12,13,14,15}

    \implicant{1}{5}
    \implicant{2}{10}

    \draw[color=black, ultra thin] (0, 4) --
        node [pos=0.7, above right, anchor=south west] {$XW$}
        node [pos=0.7, below left, anchor=north east] {$ZY$} 
        ++(135:1);
\end{karnaugh-map}
\caption{B}
\label{fig:kmapB-4-6}
\end{subfigure}
\begin{subfigure}[b]{0.3\textwidth}
	\centering
\begin{karnaugh-map}[4][4][1][][]
    \minterms{3,4,5,6}
    \maxterms{0,1,2,7,8,9}
    \indeterminants{10,11,12,13,14,15}

    \implicant{4}{5}    
    \implicant{3}{3}
    \implicantedge{4}{4}{6}{6}   

    \draw[color=black, ultra thin] (0, 4) --
        node [pos=0.7, above right, anchor=south west] {$XW$}
        node [pos=0.7, below left, anchor=north east] {$ZY$} 
        ++(135:1);
\end{karnaugh-map}
\caption{C}
\label{fig:kmapC-4-6}
\end{subfigure}
\begin{subfigure}[b]{0.3\textwidth}
	\centering
\begin{karnaugh-map}[4][4][1][][]
    \minterms{7,8}
    \maxterms{0,1,2,3,4,6,5,9}
    \indeterminants{10,11,12,13,14,15}

    \implicant{7}{15}    
    \implicantedge{12}{8}{14}{10}   

    \draw[color=black, ultra thin] (0, 4) --
        node [pos=0.7, above right, anchor=south west] {$XW$}
        node [pos=0.7, below left, anchor=north east] {$ZY$} 
        ++(135:1);
\end{karnaugh-map}
\caption{D}
\label{fig:kmapD-4-6}
\end{subfigure}
\caption{Counting 0-9}
\label{fig:0-9}
\end{figure*}

%Figs. \ref{fig:kmapA-4-6}-\ref{fig:kmapD-4-6} 
\subsection{Displays 3 and 5}
These displays count from 0-5.  
The truth table 
%is available in 
%Table 
%\ref{tab:5}
and the corresponding K-maps using dont cares for the output variables are given in 
Fig. \ref{fig:0-5},
%Figs. \ref{fig:kmapA-3-5}-\ref{fig:kmapC-3-5} 
along with the corresponding logic equations.
    $D = 0$ in this case.
\begin{figure*}[p]
    \centering

    % ------------------ Truth Table ------------------
    \begin{tabular}{|c|c|c|c||c|c|c|c|}
\hline
Z & Y & X & W & D & C & B & A\\
\hline
0 & 0 & 0 & 0 & 0 & 0 & 0 & 1\\
0 & 0 & 0 & 1 & 0 & 0 & 1 & 0\\
0 & 0 & 1 & 0 & 0 & 0 & 1 & 1\\
0 & 0 & 1 & 1 & 0 & 1 & 0 & 0\\
0 & 1 & 0 & 0 & 0 & 1 & 0 & 1\\
0 & 1 & 0 & 1 & 0 & 0 & 0 & 0\\
\hline
\end{tabular}


    % ------------------ K-Maps (1x3 layout) ------------------
    \vspace{0.5cm}

    \begin{subfigure}[b]{0.3\textwidth}
        \centering
        \begin{karnaugh-map}[4][4][1][][]
            \maxterms{1,3,5}
            \minterms{0,2,4}
            \indeterminants{6,7,8,9,10,11,12,13,14,15}

            \implicantedge{0}{8}{2}{10}
            \draw[color=black, ultra thin] (0, 4) --
                node [pos=0.7, above right, anchor=south west] {$XW$}
                node [pos=0.7, below left, anchor=north east] {$ZY$} 
                ++(135:1);
        \end{karnaugh-map}
        \subcaption*{(A)}
        \label{fig:kmapA-3-5}
    \end{subfigure}
    \hfill
    \begin{subfigure}[b]{0.3\textwidth}
        \centering
        \begin{karnaugh-map}[4][4][1][][]
            \maxterms{0,4,3,5}
            \minterms{1,2}
            \indeterminants{6,7,8,9,10,11,12,13,14,15}

            \implicant{2}{10}
            \implicantedge{1}{1}{9}{9}

            \draw[color=black, ultra thin] (0, 4) --
                node [pos=0.7, above right, anchor=south west] {$XW$}
                node [pos=0.7, below left, anchor=north east] {$ZY$} 
                ++(135:1);
        \end{karnaugh-map}
        \subcaption*{(B)}
        \label{fig:kmapB-3-5}
    \end{subfigure}
    \hfill
    \begin{subfigure}[b]{0.3\textwidth}
        \centering
        \begin{karnaugh-map}[4][4][1][][]
            \maxterms{0,1,2,5}
            \minterms{3,4}
            \indeterminants{6,7,8,9,10,11,12,13,14,15}

            \implicant{4}{8}
            \implicant{3}{11}

            \draw[color=black, ultra thin] (0, 4) --
                node [pos=0.7, above right, anchor=south west] {$XW$}
                node [pos=0.7, below left, anchor=north east] {$ZY$} 
                ++(135:1);
        \end{karnaugh-map}
        \subcaption*{(C)}
        \label{fig:kmapC-3-5}
    \end{subfigure}

    \caption{Counting 0-5}
    \label{fig:0-5}
\end{figure*}


\subsection{Display 2 (Display1 = 2)}
This display counts from 0-3, when the first display shows 2.  
The truth table 
%is available in 
%Table 
%\ref{tab:6}
and the corresponding K-maps using dont cares for the output variables are given in 
Fig. \ref{fig:0-3}.
along with the corresponding logic equations.
    $C=D = 0$ in this case.
%Figs. \ref{fig:kmapA-2}-\ref{fig:kmapB-2} 
%
\begin{figure*}[p]
	\centering
	%\begin{table}[!h]
%\centering
\begin{tabular}{|c|c||c|c|c|c|}
\hline
X & W & D & C & B & A\\
\hline
0 & 0 & 0 & 0 & 0 & 1\\
0 & 1 & 0 & 0 & 1 & 0\\
1 & 0 & 0 & 0 & 1 & 1\\
1 & 1 & 0 & 0 & 0 & 0\\
\hline
\end{tabular}


%\caption{Counting 0-3}
%\captionof{table}{Counting 0-3}
%\label{tab:6}
%\end{table}
%	\begin{subfigure}[b]{0.45\textwidth}
%	\centering
%\begin{karnaugh-map}[2][2][1][][]
%    \minterms{0}
%    \maxterms{2,1}
%    \indeterminants{3}
%    \implicant{0}{0}
%    \draw[color=black, ultra thin] (0, 2) --
%        node [pos=0.7, above right, anchor=south west] {$W$}
%        node [pos=0.7, below left, anchor=north east] {$X$} 
%        ++(135:1);
%\end{karnaugh-map}
%\caption{A}
%\label{fig:kmapA-2}
%\end{subfigure}
%	\begin{subfigure}[b]{0.45\textwidth}
%	\centering
%\begin{karnaugh-map}[2][2][1][][]
%    \minterms{0}
%    \maxterms{2,1}
%    \indeterminants{3}
%    \implicant{0}{0}
%    \draw[color=black, ultra thin] (0, 2) --
%        node [pos=0.7, above right, anchor=south west] {$W$}
%        node [pos=0.7, below left, anchor=north east] {$X$} 
%        ++(135:1);
%\end{karnaugh-map}
%\caption{A}
%\label{fig:kmapA-2}
%\end{subfigure}
 \begin{subfigure}[b]{0.45\textwidth}
	\centering
\begin{karnaugh-map}[2][2][1][][]
    \minterms{0,2}
    \maxterms{3,1}

    \implicant{0}{2}
    \draw[color=black, ultra thin] (0, 2) --
        node [pos=0.7, above right, anchor=south west] {$W$}
        node [pos=0.7, below left, anchor=north east] {$X$} 
        ++(135:1);
\end{karnaugh-map}
\subcaption{$A = W_5^{\prime}$}
\label{fig:kmapA-2}
\end{subfigure}
 \begin{subfigure}[b]{0.45\textwidth}
	\centering
\begin{karnaugh-map}[2][2][1][][]
    \minterms{1,2}
    \maxterms{3,0}
    \implicant{1}{1}
    \implicant{2}{2}
    \draw[color=black, ultra thin] (0, 2) --
        node [pos=0.7, above right, anchor=south west] {$W$}
        node [pos=0.7, below left, anchor=north east] {$X$} 
        ++(135:1);
\end{karnaugh-map}
\subcaption{$B = W_5 X_5^{\prime} + W_5^{\prime} X_5$}
\label{fig:kmapB-2}
\end{subfigure}
\caption{Counting 0-3}
\label{fig:0-3}
\end{figure*}

\subsection{Display 1}
This display counts from 0-2, representing the first digit of the hour. 
The truth table 
%is available in 
%Table 
%\ref{tab:7}
and the corresponding K-maps using dont cares for the output variables are given in 
%Figs. \ref{fig:kmapA-2le2}-\ref{fig:kmapB-2le2}
Fig. \ref{fig:0-2},
along with the corresponding logic equations.
    $C=D = 0$ in this case.
%
\begin{figure*}[t]
    \centering
    % --- Table first ---
    \begin{center}
\begin{tabular}{|c|c||c|c|c|c|}
\hline
X & W & D & C & B & A\\
\hline
0 & 0 & 0 & 0 & 0 & 1\\
0 & 1 & 0 & 0 & 1 & 0\\
1 & 0 & 0 & 0 & 0 & 0\\
\hline
\end{tabular}
\end{center}

    \vspace{0.5cm} % spacing between table and Kmaps

    % --- K-map subfigures side by side ---
    \begin{subfigure}[b]{0.45\textwidth}
        \centering
        \begin{karnaugh-map}[2][2][1][][]
            \minterms{0}
            \maxterms{2,1}
            \indeterminants{3}
            \implicant{0}{0}
            \draw[color=black, ultra thin] (0, 2) --
                node [pos=0.7, above right, anchor=south west] {$W$}
                node [pos=0.7, below left, anchor=north east] {$X$} 
                ++(135:1);
        \end{karnaugh-map}
        \caption{A}
        \label{fig:kmapA-2le2}
    \end{subfigure}
    \hfill
    \begin{subfigure}[b]{0.45\textwidth}
        \centering
        \begin{karnaugh-map}[2][2][1][][]
            \minterms{1}
            \maxterms{2,0}
            \indeterminants{3}
            \implicant{1}{1}
            \draw[color=black, ultra thin] (0, 2) --
                node [pos=0.7, above right, anchor=south west] {$W$}
                node [pos=0.7, below left, anchor=north east] {$X$} 
                ++(135:1);
        \end{karnaugh-map}
        \caption{B}
        \label{fig:kmapB-2le2}
    \end{subfigure}

    % --- Main caption for entire figure ---
    \caption{Counting 0–2}
    \label{fig:0-2}
\end{figure*}


